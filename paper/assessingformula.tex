\section{Generating The Art}

After generating all the formulae to a particular size the next step in our algorithm is
to generate all the art corresponding to each formulae as depicted in the example seen
previously. Generating all the 2x2 formulae revealed some hidden structure to our results.

\subsection{All the 2x2 Art}
\begin{figure}
\begin{center}
\begin{tabular}{r c l}
Formulae & Level & Pictures \\
\tiny{none} & 0 & empty \\
\tiny{(true), (false)} & 1 &
    \includegraphics[width=.25in]{../presentation/2x2/Shape1LVL1.png}~
    \includegraphics[width=.25in]{../presentation/2x2/Shape2LVL1.png} \\
\tiny{none} & 2 & empty \\
\tiny{($<$ x 1), ($<$ y 1), ($<$ x y), ($<$ 0 x), ($<$ 0 y), ($<$ y x)} & 3 & 
    \includegraphics[width=.25in]{../presentation/2x2/Shape1LVL3.png}~
    \includegraphics[width=.25in]{../presentation/2x2/Shape2LVL3.png}~
    \includegraphics[width=.25in]{../presentation/2x2/Shape5LVL3.png}~
    \includegraphics[width=.25in]{../presentation/2x2/Shape6LVL3.png}~
    \includegraphics[width=.25in]{../presentation/2x2/Shape3LVL3.png}~
    \includegraphics[width=.25in]{../presentation/2x2/Shape4LVL3.png}\\
\tiny{(not ($<$ x y)), (not ($<$ y x))} & 4 & 
    \includegraphics[width=.25in]{../presentation/2x2/Shape2LVL4.png}~
    \includegraphics[width=.25in]{../presentation/2x2/Shape1LVL4.png} \\
\tiny{($<$ (y + x) 1), ($<$ (y * x) 1), ($<$ 0 (y + x)), ($<$ 1 (y + x))} & 5 & 
    \includegraphics[width=.25in]{../presentation/2x2/Shape2LVL5.png}~
    \includegraphics[width=.25in]{../presentation/2x2/Shape1LVL5.png}~
    \includegraphics[width=.25in]{../presentation/2x2/Shape3LVL5.png}~
    \includegraphics[width=.25in]{../presentation/2x2/Shape4LVL5.png} \\
\tiny{none} & 6 & empty \\
\tiny{(or ($<$ y  x) ($<$ x  y))} & 7 &
    \includegraphics[width=.25in]{../presentation/2x2/Shape1LVL7.png}\\
\tiny{(not (or ($<$ y  x) ($<$ x  y)))} & 8 &
    \includegraphics[width=.25in]{../presentation/2x2/Shape1LVL8.png}
\end{tabular}
\end{center}

\caption{All the 2x2 art with its corresponding complexity.}
\label{fig:2x2}
\end{figure}

From Figure~\ref{fig:2x2}, some things are immediately obvious.  First of all,
as might match one's intuition, the all-black and all-white pictures are the
least complex pictures.  Secondly, one can see that if a formula of size $n$
exists to create a certain picture, then that picture's inverse has a
complexity of one of $n-1$, $n$, or $n+1$.  Examples of this last phenomenon
can be seen between rows 7 and 8, as well as between rows 3 and 4.  This occurs
for the simple reason that adding one symbol ({\tt not}) creates a picture's
inverse.  Also, somewhat intriguingly, we can see gaps at 2 and 6.  Although
there are formulae of size 2, none of those formulae produce a picture that has
not been already produced by formulae of a smaller size e.g (not true) , (not false).


The natural extension of our exploration of all the 2x2 artwork and the structure within the set of 2x2 artwork as a whole is the following question. What is the correlation between visual complexity (an inherently subjective notion) and
Formula complexity?  

\section(Assessing Formula Complexity)
Formula complexity is an inherent attribute of an output. It is defined as the size of the smallest formula able to produce the output in question. 
	 
	
	


