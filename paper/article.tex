\documentclass[11pt]{article}

\usepackage{url}
\usepackage{hyperref}
\usepackage{rotating}
\usepackage{array}
\usepackage{graphicx}
\usepackage[letterpaper]{geometry}

\newcommand{\twoxtwo}{$2\times2$}
\newcommand{\threexthree}{$3\times3$}

\title{Measuring the Complexity of All the Art}
\author{Jonathan Langke and Peter Boothe}
\date{\today}

\begin{document}
\maketitle

\begin{abstract}
We define formula complexity based on a restriction of formula complexity,
and measure the formula complexity of all possible \twoxtwo\ and \threexthree\
artworks.  We have also conducted a survey of the relative perceived visual
complexity of these artworks, and we report that survey's results.  We end by
discussing the links between these two notions, and find that formula
complexity does seem to be related to visual complexity.
\end{abstract}

\section{Introduction}

Art has many definitions, but the one we will use here is that a piece of black
and white digital art is a matrix where every entry is either true or false.
An artwork can be generated from a mathematical formula that evaluates to a
boolean and involves integers $x$ and $y$.  For our initial study, we
considered only \twoxtwo\ matrices, because there are only $2^{2\cdot2} = 16$
of them, and they could therefore be exhaustively generated.

The formula complexity or program-size complexity of a particular object is
the size of the smallest program which outputs that object.  This allows us to
talk about the complexity (and randomness) of individual items. Formula
complexity is, unfortunately, uncomputable.  It is, however, recursively
enumerable (RE).  Therefore, we can enumerate all well-typed expressions and
evaluate them on multiple inputs.  The size of the smallest expression which
outputs that object is then its formula complexity!

Our algorithm to generate all the art, and keep track of its formula
complexity is as follows: Enumerate all well-typed programs of size 1.  For
each enumerated formula, generate its corresponding artwork.  If this artwork
has not been previously generated, then it has Kolmogorov complexity 1.  Next,
we repeat for size 2, 3, 4, etc., until we have managed to generate all
possible \twoxtwo\ artworks.  To do this, however, we must first define what we
mean by a ``well-typed program''.

A program is well-typed if the inputs types of the functions match the type of
those function's arguments.  An example of an ill-typed program is {\tt (+
false 3)}, whereas a well-typed program is {\tt (+ 3 x)}.  We further require
that the overall type of our program be {\tt (int, int) -> bool}.  An example
of such an expression is {\tt ($\lambda$ (x y) (< x y))}, which is an
expression with two inputs, which evaluates to either true or false.

Further, we also need to define our programming language.  Our language is
constructed out of the atoms {\tt < + * x y 1 0 and or not true false}.  As a
side note, every formula in our language is also a valid Racket
program\cite{LittleSchemer}, which greatly aided us in evaluation.  


\begin{figure}
\begin{center}
\begin{tabular}{r | p{2in}   p{2in}}
Expression & (x $<$ y) &
(((1 + (1 + 1)) $<$  (x * y)) and 
          ((x + 1) $<$ (y * (1 + 1)))) \\
Expression Size & 3 symbols & 19 symbols\\
  Artwork & \includegraphics[width=1.5in]{../presentation/simple.png} &
  \includegraphics[width=1.5in]{../presentation/complex.png} \\
Formula Complexity& 3 &
19
\end{tabular}
\end{center}

\caption{Two artworks of differing formula complexity}
\end{figure}

\section{Generating The Art}

After generating all the formulae to a particular size the next step in our algorithm is
to generate all the art corresponding to each formulae as depicted in the example seen
previously. When we where able to produce all the 2x2 art we where able to garner a lot of
insight into the interrelatedness of the different artwork.

\subsection{All the 2x2 Art}
\begin{figure}
\begin{center}
\begin{tabular}{r c l}
Formulae & Level & Pictures \\
\tiny{none} & 0 & empty \\
\tiny{(true), (false)} & 1 &
    \includegraphics[width=.25in]{../presentation/2x2/Shape1LVL1.png}~
    \includegraphics[width=.25in]{../presentation/2x2/Shape2LVL1.png} \\
\tiny{none} & 2 & empty \\
\tiny{($<$ x 1), ($<$ y 1), ($<$ x y), ($<$ 0 x), ($<$ 0 y), ($<$ y x)} & 3 & 
    \includegraphics[width=.25in]{../presentation/2x2/Shape1LVL3.png}~
    \includegraphics[width=.25in]{../presentation/2x2/Shape2LVL3.png}~
    \includegraphics[width=.25in]{../presentation/2x2/Shape5LVL3.png}~
    \includegraphics[width=.25in]{../presentation/2x2/Shape6LVL3.png}~
    \includegraphics[width=.25in]{../presentation/2x2/Shape3LVL3.png}~
    \includegraphics[width=.25in]{../presentation/2x2/Shape4LVL3.png}\\
\tiny{(not ($<$ x y)), (not ($<$ y x))} & 4 & 
    \includegraphics[width=.25in]{../presentation/2x2/Shape2LVL4.png}~
    \includegraphics[width=.25in]{../presentation/2x2/Shape1LVL4.png} \\
\tiny{($<$ (y + x) 1), ($<$ (y * x) 1), ($<$ 0 (y + x)), ($<$ 1 (y + x))} & 5 & 
    \includegraphics[width=.25in]{../presentation/2x2/Shape2LVL5.png}~
    \includegraphics[width=.25in]{../presentation/2x2/Shape1LVL5.png}~
    \includegraphics[width=.25in]{../presentation/2x2/Shape3LVL5.png}~
    \includegraphics[width=.25in]{../presentation/2x2/Shape4LVL5.png} \\
\tiny{none} & 6 & empty \\
\tiny{(or ($<$ y  x) ($<$ x  y))} & 7 &
    \includegraphics[width=.25in]{../presentation/2x2/Shape1LVL7.png}\\
\tiny{(not (or ($<$ y  x) ($<$ x  y)))} & 8 &
    \includegraphics[width=.25in]{../presentation/2x2/Shape1LVL8.png}
\end{tabular}
\end{center}

\caption{All the 2x2 art with its corresponding complexity.}
\label{fig:2x2}
\end{figure}

From Figure~\ref{fig:2x2}, some things are immediately obvious.  First of all,
as might match one's intuition, the all-black and all-white pictures are the
least complex pictures.  Secondly, one can see that if a formula of size $n$
exists to create a certain picture, then that picture's inverse has a
complexity of one of $n-1$, $n$, or $n+1$.  Examples of this last phenomenon
can be seen between rows 7 and 8, as well as between rows 3 and 4.  This occurs
for the simple reason that adding one symbol ({\tt not}) creates a picture's
inverse.  Also, somewhat intriguingly, we can see gaps at 2 and 6.  Although
there are formulae of size 2, none of those formulae produce a picture that has
not been already produced by formulae of a smaller size.

Seeing all of the pictures, our hypothesis becomes a bit more obvious: What is
the correlation between visual complexity (an inherently subjective notion) and
Kolmogorov complexity?  


\section{Assessing Visual Complexity}

We constructed two websites that asked visitors to choose which of two images
they found to be of greater visual complexity.  One website asked about all the
2x2 art, and the other asked about all of the 3x3 art.  Both such sizes are
small enough to completely enumerate (there are 16 and 512 artworks in each
respective category) as well as small enough to calculate the Kolmogorov
complexity of each artwork.

We then asked users on Twitter as well as local campus students to repeatedly
assess which of two images they found more visually complex.  We treated each
assessment as if it were a competition between two players (artwork one vs
artwork two), and when the user decided upon a winner, we used the TrueSkill
algorithm to update each artwork's relative complexity based on each match's
results.


\section{Combined Results}

\begin{figure}
\includegraphics[width=6in]{3x3scatter.pdf}

\caption{Formula complexity vs visual complexity  (all points jittered by up to .25) with line of best fit.  Formula complexity has been clamped at 17.  Each data point represents one of the 512 possible \threexthree\ artworks.}
\label{fig:scatter}
\end{figure}

Once we have a measured formula complexity (Section~\ref{sec:formula}) and a
measured visual complexity (Section~\ref{sec:visual}), we plot them together.
The resulting plot may be seen in Figure~\ref{fig:scatter}.  Also included in
that plot is a line of best fit.  Visually examining the plot, we see that
there might be correlation.  Actually computing the Pearson correlation
coefficient results in a correlation coefficient of \input{3x3correlation.tex}
and an extremely low p-value ($4\cdot10^{-39}$).  All of which implies that the
two measurements are strongly correlated and it is very unlikely that this
result is due to chance.

\section{Summary and Future Work}

We defined a notion of formula complexity, which served as a ``low power''
version of Kolmogorov complexity.  We enumerated all well-typed formulae to
discover the formula complexity of a large set of very small artworks.  We then
asked a self-selected set of people to assess the visual complexity of the
artworks.  We found a nice relationship between perceived visual complexity and
formula complexity, although our results may suffer from the strong law of small
numbers\cite{smallnumbers}.

Anybody interested in reproducing or extending our results is encouraged to
fork our repository on
GitHub\footnote{\url{https://github.com/JLangke/KComplexity}} and use the code
however you see fit.  Extending our results to larger artworks would, in
particular, be quite interesting.  Discovering what languages have formula
complexity that best matches the perceived visual complexity would also be
quite interesting, as it would suggest (and perhaps help explicate) a deep link
between complexity of perception and complexity of computation.  

    \bibliographystyle{plain}
    \bibliography{bibliography} 


\end{document}
